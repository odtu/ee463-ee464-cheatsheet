\documentclass[twocolumn, ]{article}
\usepackage{amsmath}
\usepackage{graphicx}

\usepackage{geometry}
 \geometry{
 a4paper,
 total={170mm,257mm},
 left=5mm,
right=5 mm,
bottom=5 mm,
 top=5mm,
 }
 \usepackage{setspace}
\setstretch{0.3}

\begin{document}

\section*{\small Cheat Sheet for EE464}

\subsection*{\small Performance Parameters}
\begin{equation*}
True Power Factor=\frac{P}{S}=DPF \frac{I_{1,RMS}}{I_{RMS}}
\end{equation*}
\begin{equation*}
THD=\sqrt{(\frac{I_{rms}}{I_{1rms}})^2-1}
\end{equation*}



  \begin{figure}[!ht]
	\includegraphics[scale=0.30]{Fourier.png}
	\caption{Fourier Transform Table}
\end{figure}

\begin{figure}[!ht]
	\includegraphics[scale=0.20]{cuk_converter.png}
	\caption{Cuk converter}
\end{figure}

\begin{figure}[!ht]
	\includegraphics[width=2.5in,height=1in]{sepic_operation.png}
	\caption{Sepic converter}
\end{figure}


  \begin{figure}[!ht]
	\includegraphics[scale=0.30]{flybak_switch}
	\caption{Flyback switch considerations}
\end{figure}


  \begin{figure}[!ht]
	\includegraphics[scale=0.30]{pushpull}
\end{figure}

\begin{figure}[!ht]
	\includegraphics[width=2.5in,height=1in]{bipolar1.png}
	\caption{Bipolar Switching}
\end{figure}
\begin{figure}[!ht]
	\includegraphics[width=2.5in,height=1in]{unipolar1.png}
	\caption{Unipolar switching}
\end{figure}

\subsection*{Fourier Coefficients}
\begin{figure}[!ht]
	\includegraphics[width=2.5in,height=2in]{pwm_harmonics_table.png}
	\caption{Harmonics}
\end{figure}
\begin{figure}[!ht]
	\includegraphics[width=2.5in,height=1in]{unipolar_harmonics2.png}
	\caption{Unipolar Harmonics}
\end{figure}
\begin{figure}[!ht]
	\includegraphics[width=2.5in,height=1in]{bipolar_harmonics2.png}
	\caption{Bipolar Harmonics}
\end{figure}
\begin{figure}[!ht]
	\includegraphics[width=2.5in,height=2.5in]{flybackformulasfromhart.png}
	\caption{Flyback Formulas}
\end{figure}
\begin{figure}[!ht]
	\includegraphics[width=2.5in,height=1.5in]{forwardsingleswitch.png}
	\caption{Forward (single switched) Converter Formulas}
\end{figure}
\begin{figure}[!ht]
	\includegraphics[width=2.5in,height=1in]{pushpull_someformulas.png}
	\caption{Push Pull Formulas}
\end{figure}






\end{document}
